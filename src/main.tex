\documentclass[oneside, 12pt, a4paper]{book}
\usepackage{amsmath}
\usepackage{newtxtext}
%\usepackage{mathptmx} % Times New Roman Font.
\usepackage[bahasa]{babel}
\usepackage{graphicx}
\usepackage[a4paper,left=4cm,right=3cm,top=4cm,bottom=3cm,footskip=1.2cm,headsep=0.9cm,headheight=0.5cm]{geometry}
\usepackage{url}
\usepackage{hyperref}
\usepackage{setspace} % line-space-ing
\usepackage{fancyhdr}
\usepackage{lipsum}
\usepackage{etoolbox}
\usepackage{xpatch}
\usepackage{logreq}
\usepackage{csquotes}
\usepackage[table]{xcolor}
\usepackage{tabulary}
\usepackage{tikz} % render tex image from dia
\usepackage{wrapfig}  % inline image
\usepackage{subcaption}
\usepackage{indentfirst} % indent paragraph pertama
\usepackage{lipsum} % lorem ipsum

\setlength{\parindent}{2.5pc} % set indent
%\setlength{\parskip}{0.8em} % spasi antar paragraph

\definecolor{shade}{HTML}{C0C0C0}
\definecolor{shadeRGB}{RGB}{192,192,192}
\definecolor{linenum}{RGB}{233,233,233}

% syntax highlight
% section (charpter) for listing numbering
\usepackage[section, final]{minted}
\usemintedstyle{trac} %bw, colorful, igor, default, monokai, emacs, vim, tango, friendly

\usepackage[
backend=biber,
style=apa,
language=bahasa,
citestyle=authoryear,
sorting=nyt
]{biblatex}
\DeclareLanguageMapping{bahasa}{bahasa-apa}
\addbibresource{../bib/skripsi.bib}

%%%%%%%%%%%%%%%%%%%%%%%%%%%%%%%%%%%%%%%%%%%%%%%%%%%
% Patching APA biblatex style
%  
\xpatchbibmacro{labelyear+extrayear}{%
    \printtext[apadate]%
}{%
\setunit{\addperiod\space}%
\printtext%
}{}{}

\xpatchbibmacro{type+institution}{%
    \printtext[parens]%
}{%
\setunit{\addperiod\space}%
\printtext%
}{}{}

\xpatchbibmacro{type+institution}{%
       \setunit*{\addcomma\space}%
       \printlist{institution}%
       \setunit*{\addcomma\space}%
}{%
       \setunit*{\addperiod\space}%
       \printlist{institution}%
       \setunit*{\addperiod\space}%
}{}{}

\xpatchbibmacro{type+institution}{%
    \printfield{type}%
}{%
\mkbibitalic{\printfield{type}}%
}{}{}

\DeclareFieldFormat*[thesis]{title}{#1}


%%%%%%%%%%%%%%%%%%%%%%%%%%%%%%%%%%%%%%%%%%%%%%%%%%%
% Path asset
% 
\graphicspath{ {../assets/} }

%%%%%%%%%%%%%%%%%%%%%%%%%%%%%%%%%%%%%%%%%%%%%%%%%%%
% Patching APA biblatex style
% 
\usepackage{titlesec, titletoc}% http://ctan.org/pkg/titletoc
\titlecontents{chapter}% <section-type>
	[0em]% <left>
	{}% <above-code>
	{\bfseries \chaptername\ \thecontentslabel.\quad}%<numbered-entry-format>
	{}% <numberless-entry-format>
	{\bfseries\titlerule*[1pc]{.}\contentspage}% <filler-page-format>

\titlecontents{section}
	[15mm]
	{}
	{\contentslabel{8mm}}
	{\hspace*{-9mm}}
	{\titlerule*[1pc]{.}\contentspage}
	
\titlecontents{subsection}
	[27mm]
	{}
	{\contentslabel{12mm}}
	{\hspace*{-12mm}}
	{\titlerule*[1pc]{.}\contentspage}

%%%%%%%%%%%%%%%%%%%%%%%%%%%%%%%%%%%%%%%%%%%%%%%%%%%
% Ubah ukuran font bab, subbab, dan subsubbab
% 
\titleformat{\chapter}[display]
{\normalfont\large\bfseries\centering}{ BAB \thechapter}{0em}
{}
\titlespacing*{\chapter}{0pt}{-40pt}{3\baselineskip}

\titleformat*{\section}{\normalsize\bfseries}
\titleformat*{\subsection}{\normalsize\bfseries}
\titleformat*{\subsubsection}{\normalsize\bfseries}
\titleformat*{\paragraph}{\normalsize\bfseries}
\titleformat*{\subparagraph}{\normalsize\bfseries}

%%%%%%%%%%%%%%%%%%%%%%%%%%%%%%%%%%%%%%%%%%%%%%%%%%%
% Penomerah Bab di Daftar Pustaka
% 
\renewcommand{\thechapter}{\arabic{chapter}}
\renewcommand{\thesection}{\arabic{chapter}.\arabic{section}}
\renewcommand{\bibname}{Daftar Pustaka}

\addto\captionsbahasai{%
	\renewcommand{\bibname}{Daftar Pustaka}
}

\fancypagestyle{center}{
	\fancyhf{} % clear all header and footer fields
	\fancyhead[l]{\bfseries \nouppercase \rightmark} % except the center
	\fancyfoot[c]{\thepage}
	\renewcommand{\headrulewidth}{0pt}
	\renewcommand{\footrulewidth}{0pt}}

\pagestyle{center}

%%%%%%%%%%%%%%%%%%%%%%%%%%%%%%%%%%%%%%%%%%%%%%%%%%%
% Variabel
% 
\newcommand{\judulindo}{Judul Skripsi Dalam Bahasa Indonesia dengan Metode Penulisan Ilmiah}
\newcommand{\judulinggris}{Title in English which is translated from Judul Bahasa Indonesia}
\newcommand{\nama}{NAMA SAYA}
\newcommand{\nim}{NIM}

\providecommand\phantomsection{}

%%%%%%%%%%%%%%%%%%%%%%%%%%%%%%%%%%%%%%%%%%%%%%%%%%%
% http://tex.stackexchange.com/questions/11690/putting-latex-table-legend-in-a-x-y-style/11694#11694
% 
\usepackage{diagbox}
\usepackage{makecell}


%%%%%%%%%%%%%%%%%%%%%%%%%%%%%%%%%%%%%%%%%%%%%%%%%%%
% rowstyle to change font color in table row
% http://tex.stackexchange.com/questions/26360/how-to-color-the-font-of-a-single-row-in-a-table
\usepackage{array}
\makeatletter  % changes the catcode of @ to 11
\newcommand*{\@rowstyle}{}

\newcommand*{\rowstyle}[1]{% sets the style of the next row
	\gdef\@rowstyle{#1}%
	\@rowstyle\ignorespaces%
}

\newcolumntype{=}{% resets the row style
	>{\gdef\@rowstyle{}}%
}

\newcolumntype{+}{% adds the current row style to the next column
	>{\@rowstyle}%
}
\makeatother % changes the catcode of @ back to 12
%%%

%%%%%%%%%%%%%%%%%%%%%%%%%%%%%%%%%%%%%%%%%%%%%%%%%%%
% Code listing
% 
\usepackage[most, minted, listingsutf8, breakable]{tcolorbox}
\usepackage{inconsolata}

\newtcblisting[auto counter,number within=section]{sexylisting}[2][]{
	sharp corners, 
	fonttitle=\bfseries,
	colframe=shade,
	colback=white,
	text only,
%	title={Listing \thetcbcounter: #2}
	title={#2}
}

\newtcblisting[auto counter,number within=section]{mlstlisting}[2][]{
	listing engine=minted,
	minted style=default,
	minted language=python, 
	minted options={
		fontsize=\footnotesize,
		linenos,
		numbersep=2.1mm,
		mathescape,
		baselinestretch=0.8,
		python3=true,
		tabsize=4,
		autogobble=true,
		breaklines=true,
        breakanywhere=true,
		breakbytoken=true,
		breakautoindent=true,
		resetmargins=true,
		#1
	},
	sharp corners,
	colback=white,
	colframe=shade,
	listing only,
	left=7mm,
	enhanced,
	overlay={\begin{tcbclipinterior}\fill[linenum]
			(frame.south west)
			rectangle 
			([xshift=7mm]frame.north west);
			\end{tcbclipinterior}},
%	breakable,
%	title=Script \thetcbcounter: #2
    title=#2
}
%

\newtcbinputlisting[use counter from=mlstlisting]{\mlstinputlisting}[3][]{%
	listing engine=minted,
	minted language=python,
	minted style=default,
	minted options={
		fontsize=\footnotesize,
		linenos,
		numbersep=2.1mm,
		mathescape,
		baselinestretch=0.8,
		python3=true,
		tabsize=4,
		autogobble=true,
		breaklines=true,
        breakanywhere=true,
		breakbytoken=true,
		breakautoindent=true,
		resetmargins=true,
		#1
	},
	listing file={#3},
    sharp corners, 
	colback=white,
	colframe=shade,
	fonttitle=\bfseries,
	listing only,
	left=7mm,
	enhanced,
	overlay={\begin{tcbclipinterior}\fill[linenum] 
			(frame.south west)
			rectangle 
			([xshift=7mm]frame.north west);
			\end{tcbclipinterior}},
	breakable,
	title=#2
}

% Change line number color
\renewcommand{\theFancyVerbLine}{\small\ttfamily%
	\textcolor[rgb]{0.5,0.5,0.5}{\arabic{FancyVerbLine}}%
}

%%%%%%%%%%%%%%%%%%%%%%%%%%%%%%%%%%%%%%%%%%%%%%%%%%%
% Document
% 
\begin{document}
	\begin{center}
\pagenumbering{gobble}
 % Upper part of the page
 % Title
 %\hrule\vspace{5mm}
 \noindent
 {\Large \textbf{SKRIPSI}\\[1cm] }
 {\Large 
 	\textbf{\judulindo}\\[0.3cm]
	\textbf{\textit{\judulinggris}}\\[1cm]
 }
 %\hrule
 \vspace{1cm}

 \includegraphics[width=5.5cm]{logo_akakom_transparan.png}\\[2cm]

 \begin{minipage}[b]{0.75\linewidth}
 \begin{center} \normalsize

 %\line(1,0){210}\\
 \textsc{NAMA}
 
 \textsc{NIM}

 \end{center}
 \end{minipage}

 \vspace{6\baselineskip}
 {\textbf{PROGRAM STUDI SISTEM INFORMASI}
 
 \textbf{SEKOLAH TINGGI MANAJEMEN INFORMATIKA DAN KOMPUTER}
 
 \textbf{AKAKOM}
 
 \textbf{YOGYAKARTA}
 
 \the\year{}
 }
 \end{center}



	\frontmatter
	% https://tex.stackexchange.com/questions/156903/how-to-include-image-in-background
\phantomsection
\addcontentsline{toc}{chapter}{Halaman Judul}
\makebox[0pt][l]{%
  \raisebox{-460pt}[0pt][0pt]{%
    \includegraphics[width=5in]{logo_akakom_transparan_yellow}}}%
\begin{center}
{\Large \textbf{SKRIPSI}\\[1cm] }
{\Large 
	\textbf{\judulindo}\\[0.3cm]
	\textbf{\textit{\judulinggris}}
}

\vspace{4\baselineskip}

\textbf{Diajukan sebagai salah satu syarat untuk menyelesaikan studi jenjang strata satu (S1)}

\textbf{Program Studi Sistem Informasi}

\textbf{Sekolah Tinggi Manajemen Informatika dan Komputer}

\textbf{AKAKOM}

\textbf{Yogyakarta}

\vspace{3\baselineskip}

\textit{Disusun Oleh:}

\textbf{NAMA}

\textbf{NIM}

\vspace{8\baselineskip}
{\textbf{PROGRAM STUDI SISTEM INFORMASI}
	
	\textbf{SEKOLAH TINGGI MANAJEMEN INFORMATIKA DAN KOMPUTER}
	
	\textbf{AKAKOM}
	
	\textbf{YOGYAKARTA}
	
	\the\year{}
}

\end{center}

	\phantomsection \addcontentsline{toc}{chapter}{Halaman Pengesahan}

\makebox[0pt][l]{%
  \raisebox{-460pt}[0pt][0pt]{%
    \includegraphics[width=5in]{logo_akakom_transparan_yellow}}}%

\begin{center}
	{\Large \textbf{HALAMAN PENGESAHAN} }
	
	\vspace{\baselineskip}
	
	{\Large \textbf{SKRIPSI} }
	
	\vspace{\baselineskip}
	
	{\Large \textbf{\judulindo} }
	
	\vspace{\baselineskip}
	
	\textbf{Telah dipersiapkan dan disusun oleh}
	
	\vspace{\baselineskip}
    \textbf{\nama}
	
	\textbf{\nim}
	
	\vspace{\baselineskip}
	
	Telah dipertahankan didepan Tim Penguji
	
	Pada tanggal
	
	.........
	
	\vspace{\baselineskip}
	
	\textbf{Susunan Tim Penguji}
	
	\vspace{1\baselineskip}
	\noindent\hspace*{-1cm}%
	\begin{minipage}{\dimexpr0.6\textwidth+1cm}
		\textbf{Pembimbing/Penguji}
		
		\vspace{3\baselineskip}
		
		{\textbf{\underline{Dosen 1}}}
		
		\textbf{NIP/NPP.xxxx}
	\end{minipage}%
	\begin{minipage}{\dimexpr0.5\textwidth}
		\textbf{Ketua Penguji}
		
		\vspace{3\baselineskip}
		
		{\textbf{\underline{Dosen 2}}}
		
		\textbf{NIP/NPP. xxxx}
	\end{minipage}
	
	\vspace{1\baselineskip}
	
	\begin{minipage}{\dimexpr0.6\textwidth}
		\hspace{\textwidth}
	\end{minipage}%
	\begin{minipage}{\dimexpr0.5\textwidth}
		\textbf{Anggota}
		
		\vspace{3\baselineskip}
		
		\textbf{\underline{Dosen 3}}
		
		\textbf{NIP/NPP. xxxx}
	\end{minipage}
	\vspace{\baselineskip}
	\textbf{Skripsi ini telah diterima sebagai salah satu persyaratan untuk
		memperoleh gelar Sarjana Komputer}
	
	\textbf{Tanggal ...............}
	
	\vspace{\baselineskip}
	
	\textbf{Ketua Program Studi Sistem Informasi}
	
	\vspace{3\baselineskip}
	
	\textbf{\underline{Deborah Kurniawati, S.Kom., M.Cs.}}
	
	\noindent\hspace*{-3.4cm}%
	\textbf{NIP/NPP. 051149}
	
\end{center}

	\phantomsection
\addcontentsline{toc}{chapter}{Halaman Pernyataan}
\doublespacing
\begin{center}
	\textbf{PERNYATAAN}
\end{center}
\vspace{\baselineskip}

Dengan ini saya menyatakan bahwa Laporan Skripsi ini tidak terdapat karya yang
pernah diajukan untuk memperoleh gelar Ahli Madya/kesarjanaan di suatu Perguruan Tinggi, dan sepanjang pengetahuan saya juga tidak terdapat karya atau pendapat yang pernah ditulis atau diterbitkan oleh orang lain, kecuali yang secara tertulis diacu dalam naskah ini dan disebutkan dalam daftar pustaka.

\begin{flushright}
Yogyakarta, \hspace{20pt} -\hspace{20pt} - \the\year{}

\vspace{3\baselineskip}

NAMA
\end{flushright}
\singlespacing
	\phantomsection
\addcontentsline{toc}{chapter}{Persembahan dan Motto}

\begin{center}
	\textbf{MOTTO DAN PERSEMBAHAN }
\end{center}
\vspace{\baselineskip}


	\newpage
	\phantomsection
	\addcontentsline{toc}{chapter}{Daftar Isi}
	\linespread{1.25}
	\tableofcontents
	\newpage
	\phantomsection
	\addcontentsline{toc}{chapter}{Daftar Gambar}
	\listoffigures
	
	\newpage
	\phantomsection
	\addcontentsline{toc}{chapter}{Daftar Tabel}
	\listoftables
	
	\newpage
	\phantomsection
	\renewcommand\listoflistingscaption{Daftar Listing}
	\addcontentsline{toc}{chapter}{Daftar Listing}
	\listoflistings
    
    \cleardoublepage
    \fancyhead{} %Cleans the header
    \phantomsection
\addcontentsline{toc}{chapter}{Kata Pengantar}

\begin{center}
	\textbf{KATA PENGANTAR}
\end{center}
\vspace{\baselineskip}

\begin{enumerate}
    \item 
    \item
\end{enumerate}

\begin{flushright}
    Yogyakarta, \hspace{20pt} -\hspace{20pt} - \the\year{}
    
    \vspace{3\baselineskip}
    
    \nama
\end{flushright}
 

    \phantomsection
\addcontentsline{toc}{chapter}{Abstrak}

\begin{center}
	\textbf{ABSTRAK}
\end{center}
\vspace{\baselineskip}

\lipsum[23-25]

\vspace{1\baselineskip}
\noindent Kata kunci: -,-,-.
    \phantomsection
\addcontentsline{toc}{chapter}{Abstract}

\begin{center}
	\textbf{ABSTRACT}
\end{center}
\vspace{\baselineskip}

\lipsum[26-27]

\vspace{1\baselineskip}
\noindent Keyword: -,-,-,-.
    
	%time waits for no one
	\mainmatter
	\doublespacing
	\pagestyle{plain}
	\chapter{PENDAHULUAN}
\section{Latar Belakang Masalah}

\lipsum[1]

\section{Rumusan Masalah}

\lipsum[2]

\section{Ruang Lingkup}

\lipsum[3]

\section{Tujuan Penelitian}

\lipsum[4]

\section{Manfaat Penelitian}

\lipsum[5]

\section{Sistematika Penulisan}


	\chapter{TINJAUAN PUSTAKA DAN DASAR TEORI}
\section{Tinjauan Pustaka}
\begin{table}[htpb]
	\setlength\tabcolsep{3pt}
	\caption{Data Penelitian yang Berhubungan dengan .}
	\hspace*{-1.3cm}
	\begin{tabular}{| l | p{5.5cm} | p{4.5cm} |  p{2.5cm} |}
	\hline
	\rowcolor[HTML]{C0C0C0}
	\rowstyle{\color{black}}
	\bfseries Peneliti & \bfseries Judul &\bfseries -  & \bfseries -\\    \hline
	\end{tabular}
	\hspace*{-1cm}
	\label{tab:penelitian}
\end{table}

\section{Dasar Teori}

\lipsum[7]
	\chapter{METODE PENELITIAN}
\section{Bahan dan Data}

\section{Peralatan}

\subsection{Perangkat Keras}


\section{Prosedur dan Pengumpulan Data}

\section{Analisis dan Perancangan Sistem}
\subsection{Analisis Sistem}

\subsection{Perancangan Sistem}

	\chapter{IMPLEMENTASI DAN PEMBAHASAN}
\section{Implementasi dan Uji Coba Sistem}

\subsection{Implementasi}

%# PEMBAHASAN SISTEM #
%#####################
\section{Pembahasan Sistem}

	\include{./components/bab5}
	
	
	\backmatter
	\singlespacing
	\clearpage \newpage

	\phantomsection % Biar bisa muncul di Daftar Isi

    
    \printbibliography[
    heading=bibintoc, % bibintoc adds the Bibliography to the table of contents
    title={Daftar Pustaka} % If we want to override the default title "Bibliography" 
    ]
	\clearpage
	\phantomsection
	\addcontentsline{toc}{chapter}{Lampiran}
	\appendix
	\chapter*{Lampiran} 
    
\end{document}
